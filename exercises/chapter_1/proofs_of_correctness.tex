\subsection{Proofs of Correctness}

\subsubsection*{\textbf{1-7.} \emph{Prove the correctness of the following recursive algorithm to multiply two
natural numbers, for all integer constants $c\geq 2$:
\texttt{
\begin{align*}
&\text{\emph{function} M(y,z) } \\
&\text{1. If z = 0 then return(0) else} \\
&\text{2. return (cy, $\lfloor$(z/c)$\rfloor)$ + y(z mod c))}
\end{align*}
}
}}%
\textbf{case 1: \emph{c = z}}
\begin{soleqo}
M(y,z) =& \;M(cy, \lfloor(z/c)\rfloor) + y(z \;mod\; c) \\
			  =& \;M(zy, \lfloor(z/z)\rfloor) + y(z \;mod\; z) \\
			  =& \;M(zy, 1) + y(z\; mod\; z) \\
			  =& \;M(zy, 1) \\ 
			  =& \;M(czy, \lfloor(1/c)\rfloor) + zy(1 \;mod\; c) \\ 
			  =& \;M(zzy, \lfloor(1/z)\rfloor) + zy(1 \;mod\; z) \\ 
			  =& \;M(zzy, 0) + zy(1 \;mod\; z) \\ 
			  =& \;yz 
\end{soleqo}
%
\textbf{case 2: \emph{c > z}}
\begin{soleqo}
M(y,z) =& \;M(cy, \lfloor(z/c)\rfloor) + y(z \;mod\; c) \\
			  =& \;M(cy, 0) + y(z \;mod\; c) \\
			  =& \;M(cy, 0) + y(z \;mod\; c) \\
			  =& \;yz 
\end{soleqo}
%
\textbf{case 3: \emph{c < z}}

\textbf{Assumptions: } 
\begin{soleqo}
	&c \ge 2 \\
	&z \leq n \text{ (inductive hypothesis) }\\
	&y \geq 0 
\end{soleqo}

\textbf{Base Case: }
\begin{soleqo}
 z = 0, \; M(y,0) = 0 \text{, (which is true)}
\end{soleqo}

\textbf{Lemma:} 

we show that
\begin{soleqo}
	\lfloor(z/c)\rfloor*c + (z \;mod\; c) = z
\end{soleqo}

by the quotient remainder theorem 
%
\begin{soleqo}
	z =& \; cq + r \\
	  =& \; cq + z \;mod\; c \\ 
	  =& \; \lfloor(z/c)\rfloor*c + (z \;mod\; c) \;\; \text{ (1*)}
\end{soleqo}

Assuming the algorithm holds for all numbers $\leq n$  , we must show that \\
\begin{soleqo}
	M(y, n + 1) = y(n+1)
\end{soleqo}

Now,   \\
\begin{soleqo}
	M(y, n + 1) = M(cy, \lfloor((n+1)/c))\rfloor + y((n+1) \;mod\; c) 
\end{soleqo}

since \\ 
\begin{soleqo}
	c \geq& 2, \\
	\lfloor((n+1)/c)\rfloor <& n + 1 
\end{soleqo}

$\therefore$ the first term returns a valid result (based on our inductive hypothesis) so following the algorithm: \\
\begin{soleqo}
	  M(y, n + 1) =& \;  M(cy, \lfloor((n+1)/c))\rfloor + y((n+1) \;mod\; c)
\end{soleqo}

for simplicity let  $z^{'} = n+1$ \\
\begin{soleqo}
	 =& \;  cy\lfloor((z^{'})/c)\rfloor + y((z^{'}) \;mod\; c)   \\
     =& \; y(c\lfloor(z^{'}/c)\rfloor + (z^{'} \;mod\; c))  
\end{soleqo}

and from (1*) \\
\begin{soleqo}
	 =& \;  yz^{'} \;\;\blacksquare
\end{soleqo}

\subsubsection*{\textbf{1-8.} \emph{Prove the correctness of the following algorithm for evaluating a polynomial: 
$$P(x) = a_{n}^{n} + a_{n-1}^{n-1} + ... +a_{1} + a_{0}$$
\texttt{\begin{align*}
&\text{\emph{function} horner(A, x):}\\  
&\text{p =  A\textsubscript{n} }\\  
&\text{for i from n - 1 to 0 }\\
&\text{p = px + A\textsubscript{i}}\\
&\text{return  p}\\ 
\end{align*}
}
}}


\subsubsection*{\emph{For polynomials of degree 0, $P(x) = A_{0}$, which the algoritm satisfies. Assuming the algorithm holds for polyomials of degree  $\leq n$ and that A is an ordered set of coefficients of size  $n, \; [A_{n}, A_{n-1}, ... + A_{0}] $. \\}}
%
\begin{soleqo}
	 horner(A,x) = P(x) = a_{n}^{x^n} + a_{n-1}x^{n-1} + ... + a_{1}x + a_{0} 
\end{soleqo}

we must show it holds for polynomials of degree $n+1$, ie: $A^{'}$ is an ordered set of coefficients of size $n + 1$, \; ( $[A_{n+1}, A_{n}, ... A_{0}]$ )\\
%
\begin{soleqo}
&horner(A^{'}, x) :\\
&p \Rightarrow \\
&A^{'}_{n+1} \\
&A^{'}_{n+1}*x + A^{'}_{n} \\
&A^{'}_{n+1}*x^{2} + A^{'}_{n}*x + A^{'}_{n-1} \\
&. \\
&. \\
&. \\
&A^{'}_{n+1}*x^{n+1} + A^{'}_{n}*x^{n} + ... + A^{'}_{1}*x + A^{'}_{0} \\
&= \\
&A^{'}_{n+1}*x^{n+1} + horner(A, x) \;\;\blacksquare
\end{soleqo}



\subsubsection*{\textbf{1-9.} \emph{Prove the correctness of the following sorting algorithm:
\texttt{\begin{align*}
&\text{\emph{function} bubblesort(A : list[1 ...n])}  \\ 
&\text{var  int i, j }\\
&\text{for i from n to 1 } \\
&\text{for j from 1 to i - 1}  \\
&\text{if A[j] > A[j + 1] } \\
&\text{swap the values of A[j] and A[j + 1] } \\
\end{align*}
}
}}


\textbf{Our Base Case} is a list of two elements $[a,b]$ (we are indexing from 1 not 0 as usual). The base case itself has several cases:

\textbf{case 2: \emph{a > b}}
\begin{soleqo}
	&i = 2 \\
	&j = 1 \\
	&\text{if }  (A[1] > A[2]) = \;true \\
	&\text{swap values, so } A = [b,a] \\
	&i = 1 \\
	&j = 1 \\
	&\text{if } (A[1] > A[2]) = \;false \\
	&\text{so } A = [b,a] \text{ and from our assumption that a > b for case 1 the algorithm is true.} 
\end{soleqo}

\textbf{case 2: \emph{a < b}}
\begin{soleqo}
	&i = 2 \\
	&j = 1 \\
	&\text{if }   (A[1] > A[2]) = \;false \\
	&\text{swap values, so } A = [b,a] \\
	&i = 1 \\
	&j = 1 \\
	&\text{if } (A[1] > A[2]) = \;false \\
	&\text{so } A = [a,b] \text{ and from our assumption that a < b for case 2 the algorithm is true.} 
\end{soleqo}

\textbf{case 3: \emph{a = b}}
{\color{answer}{}
\begin{center}
The list is considered sorted regardless of how the algorithm may rearange its items.
\end{center}
}

Now we procede with \textbf{induction}, assuming that for all lists of size $\leq n$ the algorithm returns a sorted list We must show that for a list of size $n+1$, the algorithm returns an ordered list.
\texttt{
\begin{soleqo}
&\text{\emph{function} } bubblesort(A : list[1 ...n, n+1])  \\
&\text{var int i, j}\\
&\text{for i from n+1 to 1 }\\
&\text{for j from 1  to i - 1 }\\
&\text{if A[j] > A[j + 1]} \\
&\text{swap the values of  A[j] and A[j + 1] }
\end{soleqo}
}
{\color{answer}{}
The inner loop "bubbles up" the largest num to position i on the last j iteration of the first i iteration the altered set has $n+1$ elems with the largest element in position n+1 now we have sort the rest of the list which is of size n. Our inductive hypothesis states we can do this. $\;\; \blacksquare$}

% \texttt{
% $$\text{\emph{function} } bubblesort(A : list[1 ...n, n+1]) $$ 
% $$\text{var  int } i,\; j$$
% $$\text{for } i \;from\; n+1 \;to\; 1 $$
% $$\text{for } j \;from\; 1  \;to\; i - 1 $$
% $$\text{if } A[j] > A[j + 1] $$
% $$\text{swap the values of } A[j] \;and\; A[j + 1] $$
% }
