\subsection{Reasoning about Correctness}

\subsubsection{Expressing Algorithms}
%
\noindent\fbox{\parbox{\textwidth}{%
\emph{Take-Home Lesson: } The heart of any algorithm is an \emph{idea}. If your idea is not clearly revealed when you express an algorithm, then you are using too low-level a notation to describe it.
}%
}

\subsubsection{Problems and Properties}
%
\noindent\fbox{\parbox{\textwidth}{%
\emph{Take-Home Lesson: } An important and honorable tecnique in algorithm design is to narrow the set of llowable instances until there \emph{is} a correct and efficient algorithm. For example, we can restrit a graph problem from general graphs down to trees, or a geometric problem from two dimensions down to one.
}%
}

\subsubsection{Demonstrating Incorrectness}

\begin{itemize}
\itemsep1pt\parskip0pt\parsep0pt
\item
  \emph{Verifiability}
\item
  \emph{Simplicity}
\item
  \emph{Think small}
\item
  \emph{Think exhaustively}
\item
  \emph{Hunt for the weakness}
\item
  \emph{Seek extremes}
\end{itemize}

\noindent\fbox{\parbox{\textwidth}{%
\emph{Take-Home Lesson: } Searching for conterexamples is the best way to disprove the correctness of a heuristic.
}%
}

\subsubsection{Induction and Recursion}

\noindent\fbox{\parbox{\textwidth}{%
\emph{Take-Home Lesson: } Mathematical induction is usually the right wy to verify the correctness of a recursive or incremental insertion algorithm.
}%
}

\subsubsection{Summations}

\begin{itemize}
\itemsep1pt\parskip0pt\parsep0pt
\item
  \emph{Arithmetic progressions}
\item
  \emph{Geometric series}
\end{itemize}