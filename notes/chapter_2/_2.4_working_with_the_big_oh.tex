\subsection{Working with the Big Oh}

\subsubsection{Adding Functions}

$$O(f(n)) + O(g(n)) \longrightarrow O(max(f(n),g(n))$$

$$\Omega(f(n)) + \Omega(g(n)) \longrightarrow \Omega(max(f(n),g(n))$$


$$\Theta(f(n)) + \Theta(g(n)) \longrightarrow \Theta(max(f(n),g(n))$$

\subsubsection{Multiplying Functions}

$$O(c*f(n))) \longrightarrow O(f(n))$$

$$\Omega(c*f(n))) \longrightarrow \Omega(f(n))$$


$$\Theta(c*f(n))) \longrightarrow \Theta(f(n))$$

\noindent\rule{\textwidth}{0.4pt}

$$O(f(n)) * O(g(n)) \longrightarrow O(f(n) * g(n))$$

$$\Omega(f(n)) * \Omega(g(n)) \longrightarrow \Omega(f(n) * g(n))$$

$$\Theta(f(n)) * \Theta(g(n)) \longrightarrow \Theta(f(n) * g(n))$$

\noindent\rule{\textwidth}{0.4pt}

\textbf{Stop and Think: Hip to the SquaresTransitive Experience} \\

\emph{Show that Big Oh relationships are transitive. That is, if $f(n) = O(g(n))$ and $g(n) = O(h(n))$, then $f(n) = O(h(n))$}