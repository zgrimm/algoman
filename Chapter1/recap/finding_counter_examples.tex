\section{Introduction To Algorithm Design}

\textbf{Finding Counter Examples}

\textbf{1-1.} \emph{Show that a + b can be less than min(a, b)}
  \begin{center}
  Let $a = -1, b = -1$ \\ 
  Then $a + b = -2, \; min(a,b) = -1$ \\
  $\therefore \exists \; a, b \in Z : a+b < min(a,b)$
  \end{center}
  

\textbf{1-2.} \emph{Show that a * b can be less than min(a, b)}
  \begin{center}
  Let $a = -1, b = 5$. \\
  Then $a*b = -5, \; min(a,b) = -1$\\
  $\therefore \exists \; a, b \in Z : a*b < min(a,b)$
  \end{center}

\textbf{1-3.} \emph{Design/draw a road network with two points a and b such that the fastest route between a and b is not the shortest route}
	\begin{center}
		\begin{tikzcd}
		A \arrow[rrd, "{D=6m, \;S=3m/s}"'] \arrow[rr, "{D=5m, \;S=1m/s}"] &  & C \arrow[rr, "{D=5m, \;S=.2m/s}"] &  & B \\
		 &  & D \arrow[rru, "{D=6m, \;S=3m/s}"'] &  & 
		\end{tikzcd}
	\end{center}
Although the distance from A to B through C is shorter than going through D, road constraints limit the time it takes making the route through D faster despite it being longer.



\textbf{1-4.} \emph{Design/draw a road network with two points a and b such that the shortest route between a and b is not the route with the fewest turns}
	\begin{center}
		\begin{tikzcd}
		 &  \arrow[d, "1"] & A \arrow[rrrrrr, "6"] \arrow[l, "1"] &  &  &  &  &  & C \arrow[dd, "2"] \\
		 \arrow[d, "1"] &  \arrow[l, "1"] &  &  &  &  &  &  &  \\
		 \arrow[rr, "2"] &  & B &  &  &  &  &  &  \arrow[llllll, "6"]
		\end{tikzcd}
	\end{center}
The route from A through C to B has only two turns but is a total length of 14 units while the direct route from A to B (the shortest) has 4 turns and is a length of 6 units. $\therefore$ The shortest route between A and B is not the route with the fewest turns.

\textbf{1-5.} \emph{\textbf{The knapsack problem is as follows:} Given a set of integers S = {s1, s2,...,sn},
and a target number T, find a subset of S which adds up exactly to T. For example,
there exists a subset within S = {1, 2, 5, 9, 10} that adds up to T = 22 but not
T = 23.
Find counterexamples to each of the following algorithms for the knapsack problem.
That is, giving an S and T such that the subset is selected using the algorithm does
not leave the knapsack completely full, even though such a solution exists.}
	
		\begin{enumerate}[label=(\alph*)]
		\itemsep1pt\parskip0pt\parsep0pt
			\item{\emph{\textbf{Put the elements of S in the knapsack in left to right order if they fit, i.e. the first-fit algorithm.}} \\
				Let $S = \{1,7,9\}, \; T = 10$
			}
			\item{\emph{\textbf{Put the elements of S in the knapsack from smallest to largest, i.e. the best-fit algorithm.}} \\
				Let $S = \{1,7,9\}, \; T = 10$
			}
			\item{\emph{\textbf{Put the elements of S in the knapsack from largest to smallest.}} \\
				Let $S = \{1,4,5,7,9\}, \; T = 19$
			}
		\end{enumerate}
	% 	\begin{center}
	% \end{center}

% ## 1-5: 
% ### The knapsack problem is as follows: 
% Given a set of integers S = {s1, s2,...,sn},
% and a target number T, find a subset of S which adds up exactly to T. For example,
% there exists a subset within S = {1, 2, 5, 9, 10} that adds up to T = 22 but not
% T = 23.
% Find counterexamples to each of the following algorithms for the knapsack problem.
% That is, giving an S and T such that the subset is selected using the algorithm does
% not leave the knapsack completely full, even though such a solution exists.*

% 	(a) Put the elements of S in the knapsack in left to right order if they fit, i.e. the first-fit algorithm.
% 		A: let S = {1,7,9}, T = 10
% 	(b) Put the elements of S in the knapsack from smallest to largest, i.e. the best-fit algorithm.
% 		A: let S = {1,7,9}, T = 10
% 	(c) Put the elements of S in the knapsack from largest to smallest.	
% 		A: let S = {1,4,5,7,9}, T = 19