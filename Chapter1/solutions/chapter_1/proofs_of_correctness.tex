\subsection{Proofs of Correctness}

\subsubsection*{\textbf{1-7.} \emph{Prove the correctness of the following recursive algorithm to multiply two
natural numbers, for all integer constants $c\geq 2$}}
$$\text{function } M(y,z)$$ 
$$\text{1. If  } z = 0 \text{ then return(0) else }$$ 
$$\text{2. return } M(cy, \lfloor(z/c)\rfloor) + y(z \;mod\; c))$$

\textbf{case 1: \emph{c = z}}
\begin{soleqo}
M(y,z) =& \;M(cy, \lfloor(z/c)\rfloor) + y(z \;mod\; c) \\
			  =& \;M(zy, \lfloor(z/z)\rfloor) + y(z \;mod\; z) \\
			  =& \;M(zy, 1) + y(z\; mod\; z) \\
			  =& \;M(zy, 1) \\ 
			  =& \;M(czy, \lfloor(1/c)\rfloor) + zy(1 \;mod\; c) \\ 
			  =& \;M(zzy, \lfloor(1/z)\rfloor) + zy(1 \;mod\; z) \\ 
			  =& \;M(zzy, 0) + zy(1 \;mod\; z) \\ 
			  =& \;yz \\
\end{soleqo}
%
\textbf{case 2: \emph{c > z}}
\begin{soleqo}
M(y,z) =& \;M(cy, \lfloor(z/c)\rfloor) + y(z \;mod\; c) \\
			  =& \;M(cy, 0) + y(z \;mod\; c) \\
			  =& \;M(cy, 0) + y(z \;mod\; c) \\
			  =& \;yz \\
\end{soleqo}
%
\textbf{case 3: \emph{c < z}}

\textbf{Assumptions: } 
\begin{soleqo}
	&c \ge 2 \\
	&z \leq n \text{ (inductive hypothesis) }\\
	&y \geq 0 \\
\end{soleqo}

\textbf{Base Case: }
\begin{soleqo}
 z = 0, \; M(y,0) = 0 \text{, (which is true)}
\end{soleqo}

\textbf{Lemma:} 

we show that
\begin{soleqo}
	\lfloor(z/c)\rfloor*c + (z \;mod\; c) = z
\end{soleqo}

by the quotient remainder theorem 
%
\begin{soleqo}
	z =& \; cq + r \\
	  =& \; cq + z \;mod\; c \\ 
	  =& \; \lfloor(z/c)\rfloor*c + (z \;mod\; c) \;\; \text{ (1*)}
\end{soleqo}

Assuming the algorithm holds for all numbers $\leq n$  , we must show that \\
\begin{soleqo}
	M(y, n + 1) = y(n+1)
\end{soleqo}
Now,   \\
\begin{soleqo}
	M(y, n + 1) = M(cy, \lfloor((n+1)/c))\rfloor + y((n+1) \;mod\; c) \\
\end{soleqo}
since \\ 
\begin{soleqo}
	c \geq& 2, \\
	\lfloor((n+1)/c)\rfloor <& n + 1 \\
\end{soleqo}
$\therefore$ the first term returns a valid result (based on our inductive hypothesis) so following the algorithm: \\
\begin{soleqo}
	  M(y, n + 1) =& \;  M(cy, \lfloor((n+1)/c))\rfloor + y((n+1) \;mod\; c)
\end{soleqo}
for simplicity let  $z^{'} = n+1$ \\
\begin{soleqo}
	  =& \;  cy\lfloor((z^{'})/c)\rfloor + y((z^{'}) \;mod\; c)   \\
     =& \; y(c\lfloor(z^{'}/c)\rfloor + (z^{'} \;mod\; c))  
\end{soleqo}
and from (1*) \\
\begin{soleqo}
	 =& \;  yz^{'} \\
\end{soleqo}
$\blacksquare$



\subsubsection*{\textbf{1-8.} \emph{Prove the correctness of the following algorithm for evaluating a polynomial:}}
$$P(x) = a_{n}^{n} + a_{n-1}^{n-1} + ... +a_{1} + a_{0}$$ 
$$\text{\emph{function} } horner(A, x):$$  
$$p = A_{n}$$  
$$\text{for } i \text{ from } n - 1 \;to\; 0$$ 
$$p = px + A_{i}$$ 
$$\text{return } p$$ 

\subsubsection*{\emph{For polynomials of degree 0, $P(x) = A_{0}$, which the algoritm satisfies. Assuming the algorithm holds for polyomials of degree  $\leq n$ and that A is an ordered set of coefficients of size  $n, \; [A_{n}, A_{n-1}, ... + A_{0}] $. \\}}

\begin{soleqo}
	 horner(A,x) = P(x) = a_{n}^{x^n} + a_{n-1}x^{n-1} + ... + a_{1}x + a_{0} \\
\end{soleqo}
we must show it holds for polynomials of degree $n+1$, ie: $A^{'}$ is an ordered set of coefficients of size $n + 1$, \; ( $[A_{n+1}, A_{n}, ... A_{0}]$ )\\

\textbf{horner(} $A^{'}, x$ ) : \\
\begin{soleqo}
&p \Rightarrow \\
&A^{'}_{n+1} \\
&A^{'}_{n+1}*x + A^{'}_{n} \\
&A^{'}_{n+1}*x^{2} + A^{'}_{n}*x + A^{'}_{n-1} \\
&. \\
&. \\
&. \\
&A^{'}_{n+1}*x^{n+1} + A^{'}_{n}*x^{n} + ... + A^{'}_{1}*x + A^{'}_{0} \\
&= \\
&A^{'}_{n+1}*x^{n+1} + horner(A, x) \\
\end{soleqo}
$\blacksquare$
\emph{??? seems circular... define what we need to show better}

 	
 		


\textbf{1-9.} \emph{Prove the correctness of the following sorting algorithm:} 
$$\text{\textbf{function bubblesort } (A : list[1 ...n])} $$ 
$$\text{var  int } i,\; j$$
$$\text{for } i \;from\; n \;to\; 1 $$
$$\text{for } j \;from\; 1  \;to\; i - 1 $$
$$\text{if } A[j] > A[j + 1] $$
$$\text{swap the values of } A[j] \;and\; A[j + 1] $$

\text{\textbf{Base case} is a list of two elements $[a,b]$ (we are indexing from 1 not 0 as usual)} 

\begin{minipage}{0.95\textwidth}
\textbf{case 1: \emph{a > b}}
\begin{center}
	$i = 2$ \\
	$j = 1$ \\
	if  $(A[1] > A[2]) = \;true$ \\
	swap values, so $A = [b,a]$ \\
	$i = 1$ \\
	$j = 1$ \\
	if $(A[1] > A[2]) = \;false$ \\
	so $A = [b,a]$ and from our assumption that $a > b$ for case 1 the algorithm is true. \\
\end{center}
\end{minipage}

\begin{minipage}{0.95\textwidth}
\textbf{case 2: \emph{a < b}}
\begin{center}
	$i = 2$ \\
	$j = 1$ \\
	if  $(A[1] > A[2]) = \;false$ \\
	swap values, so $A = [b,a]$ \\
	$i = 1$ \\
	$j = 1$ \\
	if $(A[1] > A[2]) = \;false$ \\
	so $A = [a,b]$ and from our assumption that $a < b$ for case 2 the algorithm is true. \\
\end{center}
\end{minipage}

\begin{minipage}{0.95\textwidth}
\textbf{case 3: \emph{a = b}} \\
\text{The list is considered sorted regardless of how the algorithm may rearange its items} \\
\text{Now we procede with induction, assuming that for all lists of size $\leq n$ the algorithm}  \\
\text{returns a sorted list We must show that for a list of size n+1, the algorithm returns an ordered list} \\
\text{that is function bubblesort $(A : list[1 ...n, n+1])$}
\begin{center}
	\text{var int} $i, j$ \\
	\text{for } $i \;from\; n+1 \;to\; 1$ \\
	\text{for } $j \;from\; 1 \;to\; i - 1$ \\
	\text{if } $(A[j] > A[j + 1])$ \\
	\text{swap the values of } $A[j] \;and\; A[j + 1]$ \\
\end{center}
\text{The inner loop "bubbles up" the largest num to position i}
\text{on the last j iteration of the first i iteration the altered set has n+1 elems with the largest}  \\
\text{element in position n+1 now we have sort the rest of the list which is of size n.} \\
\text{Our inductive hypothesis states we can do this. $\;\; \blacksquare$}
\end{minipage}



% \textbf{1-7.} \emph{Prove the correctness of the following recursive algorithm to multiply two
% natural numbers, for all integer constants $c\geq 2$} 
%    $$\text{function } multiply(y,z)$$ 
%    $$\text{1. If  } z = 0 \text{ then return(0) else }$$ 
%    $$\text{2. return } multiply(cy, \lfloor(z/c)\rfloor) + y(z \;mod\; c))$$

% \begin{minipage}{0.95\textwidth}

% \textbf{case 1: \emph{c = z}}
% \begin{align*}
% multiply(y,z) =& \;multiply(cy, \lfloor(z/c)\rfloor) + y(z \;mod\; c) \\
% 			  =& \;multiply(zy, \lfloor(z/z)\rfloor) + y(z \;mod\; z) \\
% 			  =& \;multiply(zy, 1) + y(z\; mod\; z) \\
% 			  =& \;multiply(zy, 1) \\ 
% 			  =& \;multiply(czy, \lfloor(1/c)\rfloor) + zy(1 \;mod\; c) \\ 
% 			  =& \;multiply(zzy, \lfloor(1/z)\rfloor) + zy(1 \;mod\; z) \\ 
% 			  =& \;multiply(zzy, 0) + zy(1 \;mod\; z) \\ 
% 			  =& \;yz \\
% \end{align*}
% \end{minipage}

% \begin{minipage}{0.95\textwidth}
% \textbf{case 2: \emph{c > z}}
% \begin{align*}
% multiply(y,z) =& \;multiply(cy, \lfloor(z/c)\rfloor) + y(z \;mod\; c) \\
% 			  =& \;multiply(cy, 0) + y(z \;mod\; c) \\
% 			  =& \;multiply(cy, 0) + y(z \;mod\; c) \\
% 			  =& \;yz \\
% \end{align*}
% \end{minipage}

% \begin{minipage}{0.95\textwidth}
% \textbf{case 3: \emph{c < z}}
% % \begin{align*}
% \begin{flushleft}
% 	\textbf{Assumptions: } \\
% 	$c \geq 2$ \\
% 	$z \leq n \text{ (inductive hypothesis) }$ \\
% 	$y \geq 0$ \\
% 	\text{} \\
% 	$\textbf{Base Case: } z = 0, \; multiply(y,0) = 0 \text{, (which is true)}$ \\ 
% 	\text{} \\
% 	\text{\textbf{Lemma:} we show that } \\
% 	$$\lfloor(z/c)\rfloor*c + (z \;mod\; c) = z $$ 
% 	\text{by the quotient remainder theorem } 
% 	\begin{align*}
% 	z =& \; cq + r \\
% 	  =& \; cq + z \;mod\; c \\ 
% 	  =& \; \lfloor(z/c)\rfloor*c + (z \;mod\; c) \;\; \text{ (1*)}
% 	\end{align*}
% 	$\text{Assuming the algorithm holds for all numbers}  \leq n  \text{, we must show that  } $ \\
% 	$$multiply(y, n + 1) = y(n+1) $$
% 	$\text{Now, }$ \\
% 	$$multiply(y, n + 1) = multiply(cy, \lfloor((n+1)/c))\rfloor + y((n+1) \;mod\; c) $$
% 	$$\text{since } c \geq 2, $$
% 	$$\lfloor((n+1)/c)\rfloor < n + 1 $$
% 	$$\therefore \; \text{  the first term returns a valid result (based on our inductive hypothesis) so following the algorithm: } $$
% 	\begin{align*}
% 	  multiply(y, n + 1) =& \;  multiply(cy, \lfloor((n+1)/c))\rfloor + y((n+1) \;mod\; c)   \\
% 					    \text{ (for simplicity let } z^{'} = n+1 \text{)} \\
% 	                     =& \;  cy\lfloor((z^{'})/c)\rfloor + y((z^{'}) \;mod\; c)   \\
% 	                     =& \; y(c\lfloor(z^{'}/c)\rfloor + (z^{'} \;mod\; c))  \\
% 	                    \text{ and from (1*) } \\
%  	                     =& \;  yz^{'} \\
% 	                     =& \;  y(n+1) \\
% 	                     \;\; \blacksquare
% 	\end{align*}
% \end{flushleft}
% \end{minipage}


% \textbf{1-8.} \emph{Prove the correctness of the following algorithm for evaluating a polynomial:}
% % \begin{flushleft}
% $$P(x) = a_{n}^{n} + a_{n-1}^{n-1} + ... +a_{1} + a_{0}$$ 
% $$\text{\emph{function} } horner(A, x):$$  
% $$p = A_{n}$$  
% $$\text{for } i \text{ from } n - 1 \;to\; 0$$ 
% $$p = px + A_{i}$$ 
% $$\text{return } p$$ 
% \text{For polynomials of degree 0, $P(x) = A_{0}$, which the algoritm satisfies.}\\
% \text{Assuming the algorithm holds for polyomials of degree  $\leq n$ } 
% \text{and that A is an ordered set of coefficients of size  $n, \; [A_{n}, A_{n-1}, ... + A_{0}] $ }\\
% $horner(A,x) = P(x) = a_{n}^{x^n} + a_{n-1}x^{n-1} + ... + a_{1}x + a_{0}$ \\
% \text{we must show it holds for polynomials of degree $n+1$, ie:}
% $A^{'}$
% \text{ is an ordered set of coefficients of size } 
% $n + 1, \; [A_{n+1}, A_{n}, ... A_{0}]$ 
% \begin{center}
% \textbf{horner(} 
% $A^{'}, x) : $\\
% $p \Rightarrow$ \\
% $A^{'}_{n+1}$ \\
% $A^{'}_{n+1}*x + A^{'}_{n}$ \\
% $A^{'}_{n+1}*x^{2} + A^{'}_{n}*x + A^{'}_{n-1}$ \\
% $. $\\
% $. $\\
% $. $\\
% $A^{'}_{n+1}*x^{n+1} + A^{'}_{n}*x^{n} + ... + A^{'}_{1}*x + A^{'}_{0}$ \\
% $=$ \\
% $A^{'}_{n+1}*x^{n+1} + horner(A, x) $\\
% \end{center}
% $\;\; \blacksquare$
% \emph{??? seems circular... define what we need to show better}

 	
 		


% \textbf{1-9.} \emph{Prove the correctness of the following sorting algorithm:} 
% $$\text{\textbf{function bubblesort } (A : list[1 ...n])} $$ 
% $$\text{var  int } i,\; j$$
% $$\text{for } i \;from\; n \;to\; 1 $$
% $$\text{for } j \;from\; 1  \;to\; i - 1 $$
% $$\text{if } A[j] > A[j + 1] $$
% $$\text{swap the values of } A[j] \;and\; A[j + 1] $$

% \text{\textbf{Base case} is a list of two elements $[a,b]$ (we are indexing from 1 not 0 as usual)} 

% \begin{minipage}{0.95\textwidth}
% \textbf{case 1: \emph{a > b}}
% \begin{center}
% 	$i = 2$ \\
% 	$j = 1$ \\
% 	if  $(A[1] > A[2]) = \;true$ \\
% 	swap values, so $A = [b,a]$ \\
% 	$i = 1$ \\
% 	$j = 1$ \\
% 	if $(A[1] > A[2]) = \;false$ \\
% 	so $A = [b,a]$ and from our assumption that $a > b$ for case 1 the algorithm is true. \\
% \end{center}
% \end{minipage}

% \begin{minipage}{0.95\textwidth}
% \textbf{case 2: \emph{a < b}}
% \begin{center}
% 	$i = 2$ \\
% 	$j = 1$ \\
% 	if  $(A[1] > A[2]) = \;false$ \\
% 	swap values, so $A = [b,a]$ \\
% 	$i = 1$ \\
% 	$j = 1$ \\
% 	if $(A[1] > A[2]) = \;false$ \\
% 	so $A = [a,b]$ and from our assumption that $a < b$ for case 2 the algorithm is true. \\
% \end{center}
% \end{minipage}

% \begin{minipage}{0.95\textwidth}
% \textbf{case 3: \emph{a = b}} \\
% \text{The list is considered sorted regardless of how the algorithm may rearange its items} \\
% \text{Now we procede with induction, assuming that for all lists of size $\leq n$ the algorithm}  \\
% \text{returns a sorted list We must show that for a list of size n+1, the algorithm returns an ordered list} \\
% \text{that is function bubblesort $(A : list[1 ...n, n+1])$}
% \begin{center}
% 	\text{var int} $i, j$ \\
% 	\text{for } $i \;from\; n+1 \;to\; 1$ \\
% 	\text{for } $j \;from\; 1 \;to\; i - 1$ \\
% 	\text{if } $(A[j] > A[j + 1])$ \\
% 	\text{swap the values of } $A[j] \;and\; A[j + 1]$ \\
% \end{center}
% \text{The inner loop "bubbles up" the largest num to position i}
% \text{on the last j iteration of the first i iteration the altered set has n+1 elems with the largest}  \\
% \text{element in position n+1 now we have sort the rest of the list which is of size n.} \\
% \text{Our inductive hypothesis states we can do this. $\;\; \blacksquare$}
% \end{minipage}

	



% case 3: a = b
% 	the list is considered sorted regardless of how the algorithm may rearange its items

% Now we procede with induction, assuming that for all lists of size <= n the algorithm returns a sorted list
% We must show that for a list of size n+1, the algorithm returns an ordered list
% that is function bubblesort (A : list[1 ...n, n+1])


% 	the inner loop "bubbles up" the largest num to position i
% 	on the last j iteration of the first i iteration the altered set has n+1 elems with the largest element in position n+1
% 	now we have sort the rest of the list which is of size n. Our inductive hypothesis states we can do this